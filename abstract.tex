% !TEX root = main.tex

{\centering 

	{\Huge Abstract}

	\vspace{1cm}
	
	Graphical passwords, like the Android Pattern Lock, are a popular security mechanism for mobile devices. The mechanism was proposed as an alternative to text-based passwords, since psychology studies have recognized that the human brain have a superior memory for remembering and recalling visual information.

	This thesis aims to explore the hypothesis that human characteristics influence users’ choice of graphical passwords. A collection of 3393 user-created patterns were analysed in order to examine the correlation between people’s choice of pattern and their characteristics, like hand size, age, gender and handedness.

	This thesis first gives a detailed summary of related research on graphical passwords. Then it shows how an online survey was used for collecting user-selected passwords and information about the respondents. Lastly, the thesis explains how the data was analysed in terms of length and visual complexity in order to gain further insight in users’ choice of passwords.

	Although the data could not provide significant evidence to accept the hypothesis, the results show that password strength significantly varies between gender, age and IT experience. Additionally, analysis of all the collected patterns shows a significant bias towards the selection of pattern starting position.


}

\clearpage


{\centering 

	{\Huge Sammendrag}

	\vspace{1cm}

	Grafiske passord, som Android Pattern Lock, er en populær sikkerhetsmekanisme for mobile enheter. Mekanismen var foreslått som et alternativ til tekstbaserte passord, siden studier innen psykologi har vist at menneskehjernen er overlegen når det gjelder å huske og å gjenkjenne visuelle inntrykk.

	Denne masteroppgaven har som mål å utforske hypotesen som påstår at menneskelige karakteristikker påvirker brukeres valg av passord. En samling av 3393 brukeropprettede passord ble analysert for å undersøke om det finnes en korrelasjon mellom menneskers valg av passord og deres karakteristikker, som håndstørrelse, alder, kjønn og håndpreferanse.

	Masteroppgaven gir først en detaljert gjennomgang av relatert forskning om grafiske passord. Deretter viser den hvordan en spørreundersøkelese på internett ble brukt for å samle inn brukervalgte passord sammen med informasjon om innsenderne. Til slutt beskriver oppgaven hvordan lengde og visuell kompleksitet på mønstrene ble analysert for å oppnå en dypere forståelse av brukernes valg av grafiske passord.

	Selv om dataene ikke kunne gi signifikante bevis for å akseptere hypotesen, viser resultatene at passordstyrken varierer betydelig mellom kjønn, alder og IT-erfaring. I tillegg viser analyse av alle innsamlede mønstre at det er en skjevfordeling i hvilke noder som blir brukt som startnoder. 
}

	
