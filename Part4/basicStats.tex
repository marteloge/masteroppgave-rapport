% !TEX root = ../main.tex
\section{Analysis of Basic Statistics} \label{sec:basicstatistics}
  
  This chapter will be a summary of the basic statistics aggregated from the collected data. Basic statistics is defined as statistics where data are only aggregated and presented as is where no other algorithm or mathematical formulas have been used.

	\subsection{Background Information} \label{sec:summaryogbasicstats}
    Background information is the data representing some fundamental aspects on the collected data. This section will go though the data represented in Table \ref{tab:respondentsBasics}, Table \ref{tab:screenlockHabits}, Table \ref{tab:country}, and Figure \ref{fig:ageDistribution} summarizing the basic statistics of the collected data. 

    Table \ref{tab:respondentsBasics} is a overview of the respondents summarizing the frequency of respondents, gender, handedness, their experience with IT and security, and reading/writing orientation. In total 802 respondents completed the whole survey by answering all questions. 11 participants started creating pattern and answering, but stopped before completion. 204 people started creating patterns and stopped before creating all three types of patterns. 81 persons entered the survey and left without leaving any information. 

    %Table: Summary of the background information
    \begin{table}[H]
      \parbox{.45\linewidth}{
        \centering
        \begin{tabular}{ l | l l }
          \hline
          \multicolumn{3}{l}{\bf Respondents} \\ \hline
          Completed survey & 802 \\
          Stoped during survey & 11 \\
          Started creating patterns & 204 \\
          Opened survey and left & 81 \\ \hline
            
          \multicolumn{3}{l}{\bf Gender} \\ \hline
          Male & 529 & 66\% \\
          Female & 278 & 34\% \\ \hline

          \multicolumn{3}{l}{\bf Handedness} \\ \hline
          Left & 97 & 12\% \\
          Right & 690 & 88\% \\ \hline

          \multicolumn{3}{l}{\bf Experience with IT/Security} \\ \hline
          Yes & 470 & 59\% \\
          No & 332 & 41\% \\ \hline

          \multicolumn{2}{l}{\bf Reading/Writing direction} \\ \hline
          Top-to-bottom & 7 & 1\% \\
          Right-to-Left & 8 & 1\% \\
          Left-to-right & 792 & 98\% \\ \hline
        \end{tabular}
        \caption{Summary of the respondents}
        \label{tab:respondentsBasics}
      }
      \hfill
      \parbox{.45\linewidth}{
        \centering
        \begin{tabular}{ l | l l }
          \hline
          \multicolumn{3}{l}{\bf Mobile Operating System} \\ \hline
          Android & 464 & 58.0\% \\
          iOS & 321 & 40.0\% \\
          Windows & 16 & 1.9\% \\
          Blackberry & 1 & 0.1\% \\ \hline
 
          \multicolumn{3}{l}{\bf Used Android Unlock Pattern} \\ \hline
          Yes & 526 & 65\% \\ 
          No & 278 & 35\% \\ \hline

          \multicolumn{3}{l}{\bf Use screenlock} \\ \hline
          Yes & 655 & 82\% \\
          No & 149 & 18\% \\ \hline

          \multicolumn{3}{l}{\bf Screenlock in use} \\ \hline
          Android Pattern Lock & 202 & 31\% \\
          4-digit PIN & 237 & 36\% \\
          Fingerprint & 116 & 18\% \\
          Password & 44 & 7\% \\
          slide-to-unlock & 28 & 4\% \\
          Other & 28 & 4\% \\ \hline
        \end{tabular}
        \caption{Screen lock habits}
        \label{tab:screenlockHabits}
      }
    \end{table}

    Out of the persons entering their gender, 66\% of the participants were male and 34\% were female. Looking at handedness of the participants, 88\% were right-handed and 12\% were left-handed. The percentage of left-handedness in the population is hard to estimate, but it is stated that about 10\% of the population are left-handed \cite{lefthandedness} that seems reasonable with the percentage of left-handed participants from the survey. The reason why the percentage is somehow higher is because the survey was sent to a group of left-handed people that can be the reason for getting 12\% instead of 10\% left-handed participants.
    
    To be able to distinguish between people with experience with IT and Security it was asked about the participants experience in the survey. The network of myself and both supervisors are heavily overrepresented by people in the field of IT and Security. People with this background will might cope with the security related questions in a different way than people without the experience with IT and security. The data shows that the majority, 59\%of the respondents, had a background within IT and Security while 41\% did not. 

    It is hard to reach people outside your own network, especially to reach groups of people with a other cultures. In the dataset it was only 2\% that had an another reading and writing direction, top-to-bottom and right-to-left, than the other participants that read and writhe from left-to-right. Countries operating with a reading and writing direction are often Arabic countries or countries located in Asia. The problem is that many of these countries have during the past years been influenced from western countries. The first problem is to reach people with other reading and writing orientation than myself because all people i know read and write from left to right. I can therefore not make any further statistical analysis comparing people with different reading and writing orientation because the number of participants with that characteristics are too low to get any significant results.

    Table \ref{tab:screenlockHabits} summarized the respondents screenlock habits looking specifically at their experience with the Android Unlock Pattern, their use of screenlock mechanisms, and which screenlock they are currently using. This will provide as a overview of how people cope with security on their smartphones. Because different mobile operating systems provides different security mechanisms, the mobile operating system are also added to the table. It is not a surprise that the majority have answered the survey using either a mobile with a Android or iOs operating system that are the most popular in the the market at this point of time. 

    A total of 65\% of the participants have used the Android Pattern Lock before. The 35\% not familiar with the Android Pattern Lock will probably have their first time using the Android Pattern Lock. There will might be many experienced people with the Android Pattern Lock but it is not given that they are still using it. There are in total 82\% of the participants not using any screenlock and 18\% not using any screenlock. Among the listed screenlocks, the majority are using 4-digit PIN, Android Pattern Lock and fingerprint. The fingerprint are only available at iPhone, while Android Pattern Lock are not allowed on iPhones.

    \clearpage  

    Figure \ref{fig:ageDistribution} shows a distribution of the respondents age. The respondents are divided into 8 intervals: under 20, 20-24, 25-29, 31-34, 35-39, 40-49, 50-59, and over 60. The three last intervals have a lower frequency of participants and have therefore not being divided further into smaller intervals. The distribution have a peak at the interval 20-24. Reasons for the skewed distribution can might be a cause of the network that received the survey. The majority of my own network are students in their twenties and can therefor introduce this skewed distribution. There are still other age intervals represented in the dataset. The higher the age, the lower the frequency of respondents. This can be the same cause as for the peak, but can also be slightly lower respondents over 40 because they do not own a smartphone, nor participate in the networks where the survey were published.

    %Figure: Age distribution
    \begin{figure}[H]
      \centering
      \includegraphics[scale=0.8]{pics/analysis/AgeDist.png}
      \caption{Age distribution}
      \label{fig:ageDistribution}
    \end{figure}

    The survey asked the participants about their country of origin. The data collection ended up with a 39 countries represented in the data set where the majority of the countries was Norway, United States of America as seen in Table \ref{tab:country}.

    This data can not be used to make any analysis on any specific county. During the data collection it was a goal to avoid a homogeneous dataset. The majority of the participants is still from Norway, but is was important to get other countries represented as well to obtain as a more heterogeneous data set. 

		%Table: Country of origin (39 countires)

   {\renewcommand{\arraystretch}{2}%

		\begin{table}[H]
	    \centering
	    \begin{tabular}{ l c | c }
	      \hline
	      \multicolumn{2}{c|}{\bf Country} & {\bf \# Respondents} \\ \hline
	      \raisebox{-.4\height}{\includegraphics[scale=0.4]{pics/flags/Norway.png}} & Norway & 517 \\ \hline
	      \raisebox{-.4\height}{\includegraphics[scale=0.4]{pics/flags/USA.png}} & United States of America & 115 \\ \hline
	      \raisebox{-.4\height}{\includegraphics[scale=0.4]{pics/flags/Germany.png}} & Germany & 33 \\ \hline
	      \raisebox{-.4\height}{\includegraphics[scale=0.4]{pics/flags/CzechRepublic.png}} & Czech Republic & 31 \\ \hline
	      \raisebox{-.4\height}{\includegraphics[scale=0.4]{pics/flags/UnitedKingdom.png}} & United Kingdom & 22 \\ \hline
	      \raisebox{-.4\height}{\includegraphics[scale=0.4]{pics/flags/Russia.png}} & Russia & 13 \\ \hline
	      \raisebox{-.4\height}{\includegraphics[scale=0.4]{pics/flags/Denmark.png}} & Denmark & 7 \\ \hline
	      \raisebox{-.4\height}{\includegraphics[scale=0.4]{pics/flags/Sweden.png}} & Sweden & 6 \\ \hline
	      \raisebox{-.4\height}{\includegraphics[scale=0.4]{pics/flags/Switzerland.png}} & Switzerland & 6 \\ \hline
	      \raisebox{-.4\height}{\includegraphics[scale=0.4]{pics/flags/Australia.png}} & Australia & 5 \\ \hline
	      \raisebox{-.4\height}{\includegraphics[scale=0.4]{pics/flags/Netherlands.png}} & Netherlands & 4 \\ \hline
	      \raisebox{-.4\height}{\includegraphics[scale=0.4]{pics/flags/Chile.png}} & Chile & 4 \\ \hline
	      \raisebox{-.4\height}{\includegraphics[scale=0.4]{pics/flags/Finland.png}} & Finland & 3 \\ \hline
	      \raisebox{-.4\height}{\includegraphics[scale=0.4]{pics/flags/Austria.png}} & Austria & 3 \\ \hline
	      \raisebox{-.4\height}{\includegraphics[scale=0.4]{pics/flags/Ukraine.png}} & Ukraine & 3 \\ \hline
	      \raisebox{-.4\height}{\includegraphics[scale=0.4]{pics/flags/China.png}} & China & 3 \\ \hline
	      \multicolumn{2}{p{8cm}|}{Afghanistan, Mexico, North Korea, Pakistan, Vietnam, Luxembourg, Ireland, Tunisia} & 2 \\ \hline
	      \multicolumn{2}{p{8cm} |}{Italy, Greece, Belgium, Indonesia, Malaysia, Bahrain, Botswana, Argentina, Singapore, japan, Canada, South Korea, Hungary, Turkey, Brazil} & 1 \\ \hline
	    \end{tabular}
	    \caption{Respondents country of origin}
	    \label{tab:country}
	  \end{table}

  {\renewcommand{\arraystretch}{1}%
    
  
  \clearpage
	\subsection{Classification of Handsize and Screensize} \label{sec:classificationhandsizescreensize}

    \subsubsection*{Classification of handsize}
    To be able to analyses if there is a correlation between handsize, screensize, and finger used on the patterns created it is important to validate the subjective parameters handsize and screensize. There is no easy way to ask a person these questions over a survey, so based on the answers I have to validate the answers and decides if they are reliable enough to be further used in a analysis. 

    Handsize should be measured correctly in cm to be able to correctly classify the handsize, but it should not be a requirement that respondents have any tools available to measure their handsize, nor can I ensure that actually measure their handsize correctly. 


		%Figure: Handsize
		\begin{figure}[H]
      \centering
      \includegraphics[width=0.8\textwidth]{pics/analysis/handsize.png}
      \caption{Handsize of all participants}
      \label{fig:handsize}
    \end{figure}

    %Table: Handsize and gender
    \begin{table}[H]
      \centering
      \begin{tabular}{ c || c | c | c | c || c }
        \hline
        & {\bf Xtra Large} & {\bf Large} & {\bf Medium} & {\bf Small} & {\bf Total}\\ \hline
        {\bf Male} & 45 (9\%) & 200 (38\%) & 246 (47\%) & 38 (7\%) & 529 \\
        {\bf Female} & 3 (1\%) & 54 (19\%) & 157 (56\%) & 64 (23\%) & 278 \\ \hline
      \end{tabular}
      \caption{Handsize and gender}
      \label{tab:HandsizeGender}
    \end{table}

    %Table: Classification of handsize
    \begin{table}[H]
      \centering
      \begin{tabular}{ c || c | c | c | c | c }
        \hline
         & {\bf XL} & {\bf L} & {\bf M} & {\bf S} & {\bf XS} \\ \hline\hline
        {\bf Male}   & XL & L  & M & S &   \\
        {\bf Female} &    & XL & L & M & S \\ \hline
      \end{tabular}
      \caption{Classification of handsize for both male and female participants}
      \label{tab:classificationhandsize}
    \end{table}

    \clearpage
    \subsubsection{Classification of screensize}

    \todo[inline, color=blue!60]{Må legge til noe oppsummerende tall som beviser at Android phones ikke kan bli klassifisert}

    Classification of screensizes are a hard job due to the problem of various screen sizes and resoulutions across mobile brands. Smartphones operating on Android operation system are known to have a high variation in screen resolutions. The data collected from the mobile survey application are collected from the client, making it impossible to obtain the correct sceeensize of the respondent because it is only possible to obtain the screensize in pixels. The physical size of a mobile screen is not correlated with the pixel size. Apple have standardized the screensizes, making it possible to obtain the physical size of the respondents using a iPhone. The standarized screensizes are shown in Figure \ref{fig:iphonescreenresolutions} and Table \ref{tab:iphonescreen}.

    %Figure: Iphone screen resolutions
    \begin{figure}[H]
      \centering
      \subfigure[iPhone 3.5"]{
        \includegraphics[scale=0.45]{pics/analysis/iphone35.png}
        \label{fig:iphone35}
      }
      \subfigure[iPhone 4"]{
        \includegraphics[scale=0.45]{pics/analysis/iphone4.png}
        \label{fig:iphone4}
      }
      \subfigure[iPhone 4.7'']{
        \includegraphics[scale=0.45]{pics/analysis/iphone47.png}
        \label{fig:iphone47}
      }
      \subfigure[iPhone 5.5'']{
        \includegraphics[scale=0.45]{pics/analysis/iphone55.png}
        \label{fig:iphone55}
      }

      \caption{iPhone screen resolutions}
      \label{fig:iphonescreenresolutions}
    \end{figure}

    There are only 4 physical screensizes used on iPhones illustrated in Figure \ref{fig:iphonescreenresolutions}. In my dataset there are only obtained the pixel sizes of the screen where 4 different types were obtained. The subjective answer and the corresponding pixel size are illustrated in Figure \ref{fig:iphoneScreenDist}. Looking at Table \ref{tab:iphonescreen} we can see that there are two resolutions that have a unique correspondence to a physical size, 320$\times$480px  and 414$\times$736px. There are two sizes that correspond to two different physical sizes because iPhone 5 and 6 can both have the screen resolutions 320$\times$568px while iPhone 6 and 6+ can both have the screen resolution 375$\times$667px. Table \ref{tab:iphonescreen} contains a purposed size for each model. There do not exist any scale defining mobile screens from S-L, so this is my own subjective classification of the screensized based on the most popular screensizes on the market. Because of the two overlaps, the 320$\times$480px can be both small or medium, and 414$\times$736px can be both medium and large.

    Comparing my own subjective meaning in Table \ref{tab:iphonescreen} and the subjective classification of the respondents in Figure \ref{fig:iphoneScreenDist}, I will try to make an conclusion if the data are valid for further use. 

    The smallest screen resolution, 320$\times$480px, do have very different subjective meaning of the screensize. Comparing the different models with that size with the other models, the screen can be classified to have a small screensize. It looks promising that none of the respondents classified 320$\times$480px as a large screen. The next screen resolution is the 320$\times$568px. The most of the participants have classified the smartphone as small, while the majority have classified the screen as medium. The only possible iPhones that can have that size is the iPhone 5 and 6 that have a medium screensize compared to other models. There are also none of the respondents classifying the screen resolution 320$\times$568px as a large screen. The next is rather hard to precisely classify because the 375$\times$667px can be both an iPhone 6 or 6+. iPhone 6 are compared to other models a medium screen while iPhone 6+ are one of the biggest screens in the market, making iPhone 6 a medium screen and iPhone 6+ as a large screen. The 3 respondents classifying the 375$\times$667px screen as small screen can be moved up to a medium screen. The last screen resolution, 414$\times$736px, can we just define as a large screensize because the only iPhone having this size is the iPhone 6+. 

    %Figure: Iphone screensize distribution
    \begin{figure}[H]
      \centering
      \includegraphics[scale=0.85]{pics/analysis/IphoneScreenDist.png}
      \caption{Iphone screensize distribution}
      \label{fig:iphoneScreenDist}
    \end{figure}

    %Table: Iphone Screen classification
    \begin{table}[H]
      \centering
      \begin{tabular}{ l | l | l | l }
        \hline
        {\bf Iphone Model}  & {\bf Resolution (px)} & {\bf Inches} & {\bf Size} \\ \hline
        iPhone 2G, 3G, 3GS, 4, 4s  &  320 $\times$ 480  &  3.5'' & S\\
        iPhone 5, 5s        &  320 $\times$ 568  &  4'' & M \\
        iPhone 6            &  320 $\times$ 568, 375 $\times$ 667  &  4.7'' & M/L \\
        iPhone 6 Plus       &  375 $\times$ 667, 414 $\times$ 736  &  5.5'' & L \\ \hline        
      \end{tabular}
      \caption{Iphone screensizes}
      \label{tab:iphonescreen}
    \end{table}

    

    \todo[inline, color=blue!60]{Legge til en konklusjon om man kan stole på resultatene (bare iphones) og om de kan brukes videre i en analyse.}
    

  \clearpage
	\subsection{Created Patterns} \label{sec:createdpatterns}

		%Table: Starting node dist
		\begin{table}[H]
      \centering
      \begin{tabular}{ c || c | c || c | c | c | c }
        \hline
        {\bf Start node} & All & 1,2,3 & 1 & 2 & 3 & T \\ \hline
        1 & 44\% & 42\% & 43\% & 41\% & 42\% & 51\% \\
        3 & 15\% & 15\% & 16\% & 14\% & 13\% & 14\% \\
        7 & 14\% & 14\% & 13\% & 15\% & 14\% & 13\% \\
        2 & 9\%  & 9\%  & 10\% & 9\%  & 8\%  & 7\%  \\
        4 & 6\%  & 7\%  & 6\%  & 7\%  & 7\%  & 6\%  \\
        5 & 4\%  & 4\%  & 4\%  & 4\%  & 4\%  & 3\%  \\
        9 & 4\%  & 4\%  & 3\%  & 3\%  & 5\%  & 3\%  \\
        8 & 2\%  & 3\%  & 2\%  & 3\%  & 3\%  & 2\%  \\
        6 & 2\%  & 2\%  & 2\%  & 2\%  & 3\%  & 2\%  \\ \hline
      \end{tabular}
      \caption{Starting node (1 = Shopping, 2 = Smartphone, 3 = Bank, T = Training)}
      \label{tab:startingNode}
    \end{table}

    %Figure: Staring node for all patterns
    \begin{figure}[H]
      \centering
      \includegraphics[scale=0.45]{pics/analysis/startingNode.png}
      \caption{Staring node for all patterns}
      \label{fig:startingNode}
    \end{figure}

    %Figure: Number of unique patterns
    \begin{figure}[H]
      \centering
      \begin{minipage}[b]{0.40\linewidth}
      \centering
        \includegraphics[scale=0.4]{pics/analysis/uniquePatternsVenn.png}
      \end{minipage}%
      \begin{minipage}[b]{0.30\linewidth}
        \centering
        \begin{tabular}{ c | c }
          \hline
          Shopping &  442 \\
          Smartphone & 419 \\
          Bank & 555 \\
          Training & 414 \\ \hline \hline
          All types & 1196 \\ \hline
        \end{tabular}
        \vspace{1cm}
      \end{minipage}
      \caption{Number of unique patterns}
      \label{fig:test}
    \end{figure}

    %Table: typing habits, handedness
    \begin{table}[H]
      \parbox{.5\linewidth}{
        \centering
        \begin{tabular}{ l | l | l }
          \hline
          {\bf Hand used} & {\bf Finger used} & {\bf \#} \\ \hline
          \multirow{3}{*}{Right hand} & Thumb & 366 \\
          & Forefinger & 41 \\
          & Other & 8 \\ \hline
          \multirow{3}{*}{Left hand} & Thumb & 33 \\
          & Forefinger & 217 \\
          & Other & 23 \\ \hline
        \end{tabular}
        \caption{{\bf Right-handed} typing habits}
        \label{tab:righthandfinger}
      }
      \hfill
      \parbox{.5\linewidth}{
        \centering
        \begin{tabular}{ l | l | l }
          \hline
          {\bf Hand used} & {\bf Finger used} & {\bf \#} \\ \hline
          \multirow{3}{*}{Right hand} & Thumb & 22 \\ 
          & Forefinger & 26 \\
          & Other & 6 \\ \hline
          \multirow{3}{*}{Left hand} & Thumb & 26 \\ 
          & Forefinger & 10 \\
          & Other & 4 \\ \hline
        \end{tabular}
        \caption{{\bf Left-handed} typing habits}
        \label{tab:lefthandfinger}
      }
    \end{table}

    %Figure: Average pattern creation time (seconds) and Pattern Length (nodes)
    \begin{figure}[H]
      \centering
      \subfigure[Average pattern creation time (seconds)]{
        \includegraphics[scale=0.53]{pics/analysis/avgCreationTime.png}
      }
      \subfigure[Average Pattern Length (nodes)]{
        \includegraphics[scale=0.53]{pics/analysis/avgPatternLength.png}
      }
      \label{fig:creationTime}
    \end{figure}

    %Figure: Pattern Length distribution
    \begin{figure}[H]
      \centering
      \subfigure{
        \includegraphics[width=0.45\textwidth]{pics/analysis/patternLength.png}
      }
      \subfigure{
        \includegraphics[width=0.45\textwidth]{pics/analysis/usedALPpatterncreationtime2.png}
      }
    \end{figure}

    %Figure: Pattern creation time and experience with ALP
    \begin{figure}[H]
      \centering
      \includegraphics[scale=0.65]{pics/analysis/usedALPpatterncreationtime.png}
      \caption{Creation time based on experience with ALP}
      \label{fig:usedALPpatterncreationtime}
    \end{figure}

    %Figure:Pattern length and type distribution
    \begin{figure}[H]
      \centering
      \includegraphics[width=\textwidth]{pics/analysis/patterntypePatternLength.png}
      \caption{Pattern length and type distribution}
      \label{fig:patternTypePatternLength}
    \end{figure}

    %Table: Top 20 patterns and number of appearances
    \begin{table}[H]
	    \centering
	    \begin{tabular}{l || l l | l l | l l | l l }
	      \hline
	      {\bf \#} & \multicolumn{2}{c|}{\bf Shopping} & \multicolumn{2}{c|}{\bf Smartphone} & \multicolumn{2}{c|}{\bf Bank} & \multicolumn{2}{c}{\bf Training} \\ \hline
	      1  & 1478      & 26 & 14789      & 27 & 14789      & 17 & 1236       & 55 \\ 
	      2  & 12369     & 24 & 1478       & 23 & 1478       & 16 & 1478       & 29 \\
	      3  & 1236      & 21 & 12369      & 22 & 1236       & 15 & 12369      & 24 \\
	      4  & 321456987 & 19 & 1235789    & 16 & 123654     & 9  & 1235789    & 19 \\
	      5  & 14789     & 18 & 1236       & 15 & 4569       & 8  & 1258       & 13 \\
	      6  & 2589      & 15 & 1596       & 12 & 12369      & 8  & 1596       & 13 \\
	      7  & 1478963   & 14 & 1258       & 8  & 123654789  & 6  & 1254       & 12 \\
	      8  & 1235789   & 10 & 4569       & 8  & 1236987    & 6  & 12589      & 12 \\
	      9  & 1258      & 9  & 15963      & 8  & 1235789    & 6  & 123698745  & 11 \\
	      10 & 147852369 & 9  & 7415369    & 8  & 14753      & 6  & 3214       & 10 \\
	      11 & 1596      & 9  & 2486       & 8  & 7536       & 6  & 1235       & 10 \\
	      12 & 3698      & 8  & 7412369    & 7  & 1258       & 5  & 14789      & 10 \\
	      13 & 2587      & 8  & 4789       & 7  & 2563       & 5  & 123654789  & 9  \\
	      14 & 74123     & 7  & 123654789  & 7  & 1256       & 5  & 12357      & 9  \\
	      15 & 15987     & 7  & 1452       & 7  & 321456987  & 5  & 1452       & 8  \\
	      16 & 12357     & 7  & 7536       & 7  & 2589       & 5  & 321456987  & 8  \\
	      17 & 1456      & 6  & 3698       & 6  & 147852369  & 5  & 7896       & 8  \\
	      18 & 123654789 & 6  & 75369      & 6  & 75369      & 4  & 1256       & 8  \\
	      19 & 7456      & 6  & 3578       & 6  & 32147      & 4  & 3256       & 7  \\
	      20 & 3215987   & 6  & 14753      & 6  & 7456       & 4  & 2486       & 7  \\ \hline
	    \end{tabular}
	    \caption{Top 20 patterns and number of appearances}
	    \label{tab:top20}
  	\end{table}

  	%Table: Top 100 pattern and number of appearances
	  \begin{table}[H]
	    \centering
	    \begin{tabular}{l | l l | l | l l  | l | l l }
	      \hline
	      {\bf 1}  & 1236       & 106 & {\bf 35} & 74269    & 17 & {\bf 69} & 145236987 & 9 \\
	      {\bf 2}  & 1478       & 94  & {\bf 36} & 7415369  & 16 & {\bf 70} & 32587 & 9 \\ 
	      {\bf 3}  & 12369      & 78  & {\bf 37} & 14569    & 16 & {\bf 71} & 2365 & 9 \\ 
	      {\bf 4}  & 14789      & 72  & {\bf 38} & 75369    & 15 & {\bf 72} & 1478965 & 9 \\ 
	      {\bf 5}  & 1235789    & 51  & {\bf 39} & 1236987  & 15 & {\bf 73} & 4268 & 9 \\ 
	      {\bf 6}  & 321456987  & 37  & {\bf 40} & 7412     & 15 & {\bf 74} & 14785 & 9 \\  
	      {\bf 7}  & 1596       & 37  & {\bf 41} & 32147    & 15 & {\bf 75} & 1569 & 8 \\ 
	      {\bf 8}  & 1258       & 35  & {\bf 42} & 7412369  & 14 & {\bf 76} & 8426 & 8 \\ 
	      {\bf 9}  & 2589       & 28  & {\bf 43} & 3214789  & 14 & {\bf 77} & 1536 & 8 \\ 
	      {\bf 10} & 123654789  & 28  & {\bf 44} & 5896     & 14 & {\bf 78} & 23698 & 8 \\ 
	      {\bf 11} & 4569       & 26  & {\bf 45} & 1598     & 14 & {\bf 79} & 14789632 & 8 \\ 
	      {\bf 12} & 147852369  & 26  & {\bf 46} & 35789    & 13 & {\bf 80} & 3521 & 8 \\ 
	      {\bf 13} & 123698745  & 23  & {\bf 47} & 1475369  & 13 & {\bf 81} & 159874 & 8 \\ 
	      {\bf 14} & 14753      & 23  & {\bf 48} & 3574     & 13 & {\bf 82} & 3215789 & 8 \\ 
	      {\bf 15} & 3698       & 22  & {\bf 49} & 3256     & 13 & {\bf 83} & 9654 & 7 \\ 
	      {\bf 16} & 1452       & 22  & {\bf 50} & 36987    & 13 & {\bf 84} & 12369874 & 7 \\ 
	      {\bf 17} & 7536       & 21  & {\bf 51} & 7456     & 12 & {\bf 85} & 321456 & 7 \\ 
	      {\bf 18} & 3214       & 21  & {\bf 52} & 123654   & 12 & {\bf 86} & 753698 & 7 \\ 
	      {\bf 19} & 7896       & 20  & {\bf 53} & 2569     & 12 & {\bf 87} & 7415963 & 7 \\ 
	      {\bf 20} & 2486       & 20  & {\bf 54} & 3215987  & 12 & {\bf 88} & 4123 & 7 \\ 
	      {\bf 21} & 1478963    & 20  & {\bf 55} & 1475     & 12 & {\bf 89} & 3258 & 7 \\ 
	      {\bf 22} & 1458       & 19  & {\bf 56} & 2369     & 11 & {\bf 90} & 42586 & 7 \\ 
	      {\bf 23} & 1456       & 19  & {\bf 57} & 325698   & 11 & {\bf 91} & 7423 & 7 \\ 
	      {\bf 24} & 12589      & 19  & {\bf 58} & 12569    & 11 & {\bf 92} & 159873 & 7 \\ 
	      {\bf 25} & 15987      & 19  & {\bf 59} & 12365    & 11 & {\bf 93} & 32147896 & 7 \\ 
	      {\bf 26} & 12357      & 19  & {\bf 60} & 2563     & 11 & {\bf 94} & 2357 & 7 \\ 
	      {\bf 27} & 15963      & 18  & {\bf 61} & 4785     & 11 & {\bf 95} & 357896 & 7 \\ 
	      {\bf 28} & 74123      & 18  & {\bf 62} & 147852   & 11 & {\bf 96} & 1523 & 7 \\ 
	      {\bf 29} & 2587       & 18  & {\bf 63} & 75321    & 10 & {\bf 97} & 3596 & 7 \\ 
	      {\bf 30} & 1254       & 18  & {\bf 64} & 4563     & 10 & {\bf 98} & 1452369 & 6 \\ 
	      {\bf 31} & 4789       & 18  & {\bf 65} & 159632   & 10 & {\bf 99} & 7453 & 6 \\ 
	      {\bf 32} & 3578       & 18  & {\bf 66} & 7586     & 10 & {\bf 100} & 8523 & 6 \\ 
	      {\bf 33} & 1256       & 18  & {\bf 67} & 741258   & 9  & & & \\ 
	      {\bf 34} & 1235       & 17  & {\bf 68} & 258963   & 9  & & & \\ \hline
	    \end{tabular} 
	    \caption{Top 100 pattern and number of appearances}
	    \label{tab:top100}
	  \end{table}

	\clearpage
	\subsection{3-gram Analysis} \label{sec:3gram}
		
		%Figure: Most common 3-gram to less common 3-gram
		\begin{figure}[H]
	    \subfigure{
	      \includegraphics[width=0.31\textwidth]{pics/analysis/3gram1.png}
	    }
	    \subfigure{
	      \includegraphics[width=0.31\textwidth]{pics/analysis/3gram2.png}
	    }
	    \subfigure{
	      \includegraphics[width=0.31\textwidth]{pics/analysis/3gram3.png}
	    }
	    \caption{Most common 3-gram to less common 3-gram}
	    \label{fig:3gram}
	  \end{figure}

	\clearpage
	\subsection{The Impact of Using Latin Square} \label{sec:latinsquareimpact}

    \todo[inline, color=orange!80]{Flytte denne til diskusjon? Kanskje sette den sammen med kontrolldataene?}

		%Figure: Percentage of times the pattern orders occurred
		\begin{figure}[H]
      \centering
      \includegraphics[scale=0.5]{pics/analysis/patternOrder.png}
      \caption{Percentage of times the pattern orders occurred}
      \label{fig:patternOrder}
    \end{figure}

  

