% !TEX root = ../main.tex
\chapter{Results}\label{chap:results}

	This chapter presents the results provided by the collected data from the survey. Section \ref{sec:basicstatistics} is a overview of the population while Section \ref{sec:findingsEntirePopulation} and \ref{sec:findingsSpecificSubgroups} looks at the results with respect to entire population and different subgroups, respectively. Section \ref{sec:preprocessvalidate} is the last section including a description of how the data were preprocessed before being analyzed. The subgroups included in Section \ref{sec:findingsSpecificSubgroups} are gender, age, handedness, experience with IT and security, hand size and screen size. For the subgroups, a simple two-tailed t-test are performed to test a significant difference in two samples looking at pattern length and visual complexity. 

	The results in Section \ref{sec:findingsEntirePopulation} and \ref{sec:findingsSpecificSubgroups} are divided into pattern creation time, pattern length, and visual complexity. The pattern creation time is the time used for creating a pattern, recorded from the start when the grid appeared until the user submitted the pattern. The pattern length are defined as the number of dots used to form a pattern. Each dot has an own sequence number and can only appear once. The minimum length of a pattern is 4 dots while the maximum length is 9 dots. The number of unique patterns of all possible pattern lengths is described in Table \ref{tab:combinations}. The pattern length are visualized in two different ways; the average pattern length and the distribution of pattern length. Pattern complexity, e.g. pattern strength, is calculated from a mathematical formula that utilizes the visual aspects of patterns. The formula and parameters used are described in detail in Section \ref{sec:alp}. In short, the formula uses the size (number of nodes), physical size (length) of the pattern, number of intersections and number of overlaps. The minimum strength is 6.340 and the maximum strength is 46.807.
