% !TEX root = ../main.tex
\section{Hypothesis}\label{sec:hypothesis}

	This section will include hypothesis to be tested. The hypothesis has its origin from the conducted literature study in Chapter \ref{chap:relatedwork}.

	{\renewcommand\labelitemi{}
		\begin{itemize}
  		\item \texttt{$H_{0}$: Human properties have no influence on users choice of \\graphical passwords}
  		\item \texttt{$H_{1}$: Users choice of graphical passwords are influenced by \\the human properties of the user}
  	\end{itemize}
  }

  The hypothesis to be tested is referred to as the null hypothesis (abbreviated $H_{0}$). We start with the assumption that the null hypothesis is true; human properties have no influence on users choice of graphical passwords. Along with $H_{0}$ it is stated an alternative hypothesis (abbreviated $H_{1}$). The alternative hypothesis states that users choice in graphical passwords is influenced by the human properties of the user. If the null hypothesis is rejected, we accept the alternative hypothesis.

  The results of this proposed research will have some limitations. {\it First}, the hypotheses do not include all potential human properties. If the results of the statistical test are accepting the alternative hypothesis, the results are not valid for proving a correlation between users choice in graphical passwords and human properties that are not selected in this research. {\it Second}, the hypothesis does neither prove that human choices based on the human properties is valid for all graphical password schemes. The selected graphical password scheme is the Android Unlock Pattern and the decision for choosing this particular scheme is described in the summary of the literature review in Section \ref{sec:resultsLiteratureStudy}. In the research strategy, there will be a description of the selected human properties (Section \todo{Check refernece!}) that this study find relevant to users choice in graphical patterns.

  The results from this study can further be used to evaluate the security of the Android Unlock Pattern. If there is a correlation between users choice in Android Unlock Patterns and the users human properties, it can be used to make dictionary attacks by predicting user's pattern locks. An ability to predict user's choice in locking patterns is a reduction in the scheme's security. 
