% !TEX root = ../main.tex
\section{Graphical Passwords and Mobile Devices}\label{sec:pwmobiledevices}

  Users are not only dependent on remembering passwords across multiple web pages and systems but do also need to remember passwords for our small mobile devices. The use of handheld devices, such as smartphones and tablets, has seen tremendous growth in the recent years. The smartphone in general has revived increased attention because of its increased capacity and its variety of use. On the first version of the smartphones, users could access their email, participate in social networks, as well as the basic features of a phone like calling and text messaging. During the past years, the gap between a desktop and a smartphone have become smaller and smaller. People today use their mobile phone for work purposes, mobile banking, and online shopping. This progression of the smartphone sets higher standards of the security on smartphones. A smartphone is a handy tool in daily life but do also contain a lot of sensitive of your private life. Mobile devices can easily be lost or stolen due to it small size, making an increased need for protecting the sensitive data from unauthorized access.
  
  The smartphone have emerged as an excellent platform for graphical passwords because its intuitive interaction with the touch screen in contrast to text-based passwords on mobile devices. Graphical passwords on mobile devices seem like a natural fit, as they often require direct manipulation of visual elements. For avoiding unwanted access on smartphones, different locking mechanisms are provided. The history of locking mechanisms was often a solution solely to prevent accidental use while current mobile phones require protection to secure the potentially vast amount of private data that we keep on our smartphones. The situation of our active use of mobile phones, as well as its well-suited platform for using a graphical password, makes authentication on mobile devices an interesting field of study.

  When looking at mobile security it's necessary to be familiar with the magnitude of mobile phone usage. As of 2014, over 90\% of American adults owned a mobile phone, whereas 58\% of American adults owned a smartphone \cite{MobileUseage}. Another 34\% of the users used their phone regularly instead of using other devices such as a desktop or laptop computer for searching on the Internet. The numbers are collected from a population only living in the USA, but still provides insightful information about the usage of mobile devices today.

  As stated earlier, as a cause of users storing sensitive information on their phones, it is important to understand the relationship between the use of security features and users risk perceptions. One aspect essential to understand is the reason people choose to use, or not use, screen locks on their smartphone. Engelman et al. \cite{Egelman} published a research paper in cooperation with Google on people's screen locking behavior and attitude towards security on their smartphones. They observed a strong correlation between the use of security features and risk perceptions. They reported that 33\% of the smartphone users were thinking about the locking mechanisms as too much of a hassle. At the same time, 26\% of the same population didn't think that someone would care about the information stored on their smartphone. Another research group studying the same topic revealed that 46.8\% of the participants agreed or fully agreed that unlocking their phone can be annoying. At the same time, 95.5\% of the respondents somewhat agreed or fully agreed that they liked the idea that their phone was protected \cite{habits3}.  Locking a smartphone are crucial, even if users do not prefer it because they think that it is annoying. A study reported that 29\% did not use any for of locking mechanisms \cite{MobileUseage} while another research stated that among 35\% of mobile users do not lock their phone \cite{Bruggen}. The number may vary, but it still highlights that the users want to be secure while at the same time do not wish to use security mechanisms. The results reported might be an indication of a trade-off between the time need to type a password and users risk perception. What is more important, time used or level of protection?

  In terms of security, it is interesting to look at the use of mobile devices and locking habits. Services that are in active use are known to be protected by a weaker password as a cause of the overhead in time spend typing the password. In 2014, a group of researchers published a field study of users (un)locking behavior \cite{habits3}. The problem observed was that that user had to unlock their screen frequently. In the field study, they found a significant overhead in the time used for unlocking their phone. The participants used on average 2.9\%, and up to 9\% in the worst case, of their time interacting with the smartphone unlocking the screen.

  It is stated that many users use their smartphones to perform tasks that involve utilization and storage of sensitive data. Smartphones today do not require their users to use any locking mechanism on their smartphone. As a cause of users tending to choose the easiest way out may result in the choice of not having any locking mechanism at all. By not using any locking mechanisms, the security risks of looking sensitive information are ignored. In a study, over 40\% of the users only used the basic {\it Slide-to-unlock} mechanism on their smartphone, as well as over 16\% did not use any locking mechanisms at all \cite{habits3}. This result highlights an a bad habit among mobile users that may have consequences. What happens if your mobile is stolen? A loss of a mobile phone is not just the cost of replacing the physical device, but also a loss of sensitive data. If the wrong person finds the device, sensitive data on the device may be lost and used for unintended purposes. In 2012, Pew Internet estimated that nearly a third of mobile users have had their mobile device stolen or lost \cite{StolenLost}. It is interesting to compare people's locking behavior towards phones that are stolen or lost. The same report also stated that 12\% of cell owners say that another person have accessed their phone, making the owners feel that their privacy being exposed to the public.

  Besides losing a physical device, what consequences relates to the loss of a smartphone? One point of attack is to get access to people's email account. If you can grant access to someone's email, you probably can get access to a lot more as a cause of password reset sent to the user's email. A study reported that all of their interviewed participants had their email account automatically logged in, as well as 31\% of them did not use any locking mechanism at all \cite{Egelman}. The same research group investigated how much information you could gain from getting access to a person email account. The results revealed that both users with or without locking mechanisms found sensitive information in their email account like SSN, Bank Account Number, Email Password and Home Address. Mobile devices might contain more sensitive information than users are aware of.
