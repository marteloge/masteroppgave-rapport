% !TEX root = ../main.tex
\section{Graphical Passwords and Mobile Devices}\label{sec:pwmobiledevices}

	Users are not only dependent on remembering passwords across multiple web pages and systems, but do also need to remember passwords for our small mobile devices. The use of handhold devices, such as smartphones and tablet, has seen tremendous growth in the recent years. The smartphone in general has revived increased attention because of its increased capacity and its variety of use. The first smartphones could access your email, you social network, as well as the basic features of a phone like calling and text messaging. During the past years, the gap between a desktop and a smartphone have become smaller and smaller. People today use their mobile phone for work purposes, mobile banking, online shopping, and other high security tasks. This evolution of the smartphone sets higher standards of security on your smartphone. The smartphone is a handy tool in daily life, but do also contain a lot of sensitive of your private life. Mobile devices can easily be lost or stolen due to it small size, therefore, it is a need to protect the sensitive data from unauthorized access. 
  
  The smartphone have emerged as an excellent platform for graphical passwords because it is easier to input on touchscreen as a contrast to text-based passwords. Graphical passwords on mobile devices seem like a natural fit, as they often require direct manipulation of visual elements. To avoid unwanted access, smartphones offer different locking mechanisms. The history of locking mechanisms was often a solution solely to prevent accidental use, while current mobile phones require protection in order to secure the potentially vast amount of private data that we keep on our smartphones. The situation of our rapid use of mobile phones, as well as it well-suited platform for graphical password, makes authentication on mobile devices an interesting field of study.

  When looking at mobile security it essential to be familiar with the magnitude of mobile phone usage. As of 2014, over 90\% of American adults owns a mobile phone, whereas 58\% of American adults owns a smartphone \cite{MobileUseage}. Another 34\% of the users used their phone mostly to go online instead of using other devices such as a desktop or laptop computer. This is numbers from USA, but it still provides insight information about the use of mobile phones today.

  % Awareness about the sensitivity of the data stored mobile phones
  As stated earlier, smartphone user's tends to store sensitive information on their phones, it is important to understand the relationship between the use of security features and users risk perceptions. One of the key security aspects on mobile phones that are important to understand is why people use or not use locking mechanisms on their smartphone. Engelman et al. \cite{Egelman} published a research paper in cooperation with Google on people's smartphone locking behavior and attitudes towards security of their smartphone data. They observed a strong correlation between the use of security features and risk perceptions. They reported that 33\% of the smartphone users were thinking about the locking mechanisms as too much of a hassle, while 26\% of the user's didn't think that someone would care about the information stored on their phone. Other reported results have covered that the 46.8\% of the participants agreed or fully agreed that unlocking their phone can be annoying. At the same time, 95.5\% of the users somewhat agreed or fully agreed that they liked the idea that their phone was protected \cite{habits3}. The study reported that 29\% did not lock their smartphones \cite{MobileUseage} while another research stated that among 35\% of mobile users do not lock their phone \cite{Bruggen}. The number may vary because of the background and experiences with security while the number remains quite high. This highlights that the users want to be secure, but there might be a trade-off between the time used to unlock the smartphone vs. the security risk.

  % The time used on unlocking the phone
  In terms of security, it is interesting to look at the use of mobile devices and look at the locking habits among users on mobile devices. It is known that services that are rapidly used have weaker password because of the overhead the user needs to spend on typing their password. In 2014 a group of researchers published a field study of smartphone (un)locking behavior \cite{habits3}. Some of the problems with smartphone users tends to be their rapid use of their phone. When the device is rapidly used, it results in much time unlocking their phone between every use. In the study they found that there was a significant overhead in the time used for unlocking their phone, where the users participated in the field study used 2.9\% (9\% in the worst case) of their time unlocking their smartphone.

  It is stated that many users use their smartphones to perform tasks that involve utilization and storage of sensitive data. Smartphones in use today do not require their users to have a locking mechanism on their smartphone. It is well known that users tends to choose the easiest way out and may result in the choice of not having any locking mechanism at all. Based on the outcome of the overhead in time used for unlocking their phone, the result may be to take the easiest way out by ignoring the vulnerability of not using a locking mechanism at all. It has been discovered that over 40\% of the users only used a basic ``slide-to-unlock'' mechanism on their smartphone, as well as over 16\% did not use any locking mechanisms at all \cite{habits3}. This highlights an important bad habit among mobile users. What happens if your mobile is stolen? A loss of a mobile phone is not just the cost of replacing the phone, but also a loss of sensitive data. If the wrong persons find the phone, the sensitive data on the phone may be lost and used for unintended purposes. A 2012 report from Pew Internet estimated that nearly a third of mobile users have had their device stolen or lost \cite{StolenLost}. It is interesting to comparing people's locking behavior towards phones that are stolen or lost. The same report also stated that 12\% of cell owners say that another person have accessed their phone, making the owners feel that their privacy have been exposed to the public.

  Besides losing a physical device, what consequences are users exposed to? One point of attack is to get access to a people's email. If you can grant access to someone's email, you probably can get access to a lot more. In a study reported that all of their interview participants had their email account automatically logged in, as well as 31\% of them did not use any locking mechanism at all \cite{Egelman}. The same research group investigated how much information you could gain from getting access to a person email account. The results showed that both users with or without locking mechanisms found sensitive information in their email account like SSN, Bank Account Number, Email Password and Home Address.
