% !TEX root = ../main.tex
\section{Summary of Related Work}\label{sec:resultsLiteratureStudy}
    
  In the literature review, I have reviewed published research that this project find relevant when looking at graphical passwords. The literature review started with a review of graphical password from a historical point of view. This provides an overview of the purposed schemes with the aim of understanding the reason they were proposed. Each scheme is attempting to either improve drawbacks with earlier published schemes while some schemes just try to create something new.

  When going through the history of the published schemes, it was discovered a trade-off between usability and security. Some graphical password schemes solely look at a new way of designing a password scheme by looking at the usability, the security, and sometimes both.

  When looking for text-based passwords and PINs, it is often observed that the passwords are more than just numerical values and letters. I believe that we can look at graphical passwords in the same way; graphical passwords are more than just images and graphical elements. This is the reason I added a section about psychology and human factors in Section~\ref{sec:humanfactors} in order to understand user's choice in graphical passwords. When looking back at Section~\ref{sec:usability}, user selects passwords that they can recall by associate the selected password with something they know or are.

  What is not yet answered in the domain of graphical passwords? Of the published research that included the keyword ``human factors'', only a few of them actually looked at users choice in graphical passwords based on who the users are. Most of them only looked at user's selection of passwords in general.

  Smartphones are in rapid use and is a part of many users everyday life used to perform their daily tasks. The smartphone is also an interesting platform for graphical password because it of the touch sensitive screen. When conducting research, I wanted to look further into a graphical password scheme that is in use. The Android Unlock pattern is a well-known graphical password scheme used on smartphones. As this research is familiar with, there is only published one paper looking at the schemes security \cite{Uellenbeck}. As stated, I believe that graphical passwords are more than just images and graphical elements. The research that inspired me only looked at what patterns users in general selected. Are we able to add a new dimension to an analysis of users choice in graphical password based on who they are? 
  