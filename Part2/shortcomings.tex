% !TEX root = ../main.tex
\section{Shortcomings With Text-Based Authentication} \label{sec:shortcomings}

  User authentication is a central part of security systems. Despite the extensive number of options for authentication, text-based passwords remain the most common authentication scheme. Text-based authentication is the authentication scheme widely adopted because it is easy and inexpensive to implement, and users are familiar with the scheme. Text-based authentication also avoids the privacy issues raised by the use of biometric authentication, as well as preventing the need for a physical security device used in token-based authentication schemes. However, text-based authentication suffers from both security and usability disadvantages. As users need to remember an increasingly number of passwords, users adopt bad password habits. The term {\it habit} is often a bad thing when talking about security. A habit is often hard to change and are often a behavior that are predictable because it occurs in the same situations over and over again. 

  Password reuse is one of the known password habits among users as a cause of human limitations to be able to remember a text-based password. An another habit introduced as a cause of dealing with the problem of remembering passwords is to create short and meaningful passwords that are easier to remember. However, a consequence of creating short passwords is a vulnerability for brute-force attacks, introducing a security risk. Also having an increasingly number of accounts requires users to use a set of different passwords across multiple devices. The problem is not just to remember all the password needed, but also remembering which passwords are belonging to which account or device. The increased number of accounts and devices is an another cause for users to reuse passwords across multiple accounts and devices.

  One of the first large-scale studies on web password habits was conducted in 2007 by Microsoft Research \cite{habits1}. They analyzed text-based passwords used by 544960 Internet users over a period of 3 months. For collecting the password, Microsoft used a Windows Live Toolbar observing activities like login frequency. They were also able to see how many unique passwords the users had and how the passwords were used across separate URLs. Microsoft observed that a typical user have an average of 7 distinct passwords. Out of the seven unique passwords, five of them was re-used on different web pages. An estimate of the average number of accounts per user was estimated to 25 accounts per user.

  Password schemes have what is called a theoretical password space that is the possible combinations of passwords that a user can make. When creating a password, research have reported that use do not use the entire password space and uses only a subset of the possible passwords. The password space in use can bee seen as the practical password space, making the practical password space less than the theoretical password space (Figure \ref{fig:memorable}).  The selected passwords tell that the security of a password scheme relates to its practical password space rather than its theoretical password space.

    \begin{figure}[H]
      \centering
      \includegraphics[scale=0.55]{pics/review/EmpiricalVsPractical.png}
      \caption{Theoretical password space vs. Password space in practice}
      \label{fig:memorable}
    \end{figure}

  In a case study of 14.000 Unix passwords, a research group found that 25\% of the passwords were a group of words forming a dictionary of $3\times10^{6}$ words \cite{UnixPasswords}. This dictionary shows that an attacker can have a relatively high success rate for an attack, despite the fact that there a roughly $2\times10^{14}$ 8-character passwords consisting of digits, and upper and lower case letters. As a cause of people choosing weak passwords that are easier to remember, a significant number of user-chosen passwords falls into a small dictionary, e.g. the password space in practice \cite{Tao}. A well-designed dictionary considers to be a tiny subset of the full password space, e.g. theoretical password space, which further can be prioritized according to the likelihood for a password to be chosen. It is, therefore, commonly stated that the security of a password scheme is related to the size of its password space in practice, rather than its theoretical password space. The high success rate of dictionary attacks against text-based passwords is considered to be a cause of the recall capabilities of humans and how they choose their passwords.

  As a cause of the shortcomings with text-based authentication, graphical authentication are getting increased attention as an alternative to text-based authentication. Graphical passwords are trying to help the users to be able to create secure passwords that also are easy to remember. Instead of using text and number, graphical passwords are using images and visual objects in the authentication process. When comparing the use of text against the use of visual objects, the human brain is more capable of remembering images than text \cite{DeAngeli}. As a cause of humans being more capable of recognizing images, users will be more capable of creating more complicated passwords that are harder to guess.

  The next section will look further into the history of graphical passwords; when did it all start and how is the situation today?
  