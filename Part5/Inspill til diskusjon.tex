
  \section{Age and choice in graphical passwords}
        -- Signifikant kompleksitet for shopping og bank, ikke signifikant for smartphone
        -- Signifikant lengde for shopping og bank, ikke signifikant for mobil
        -- Forskjellig erfaring
        -- 

    \section{Gender and choice in graphical passwords}
        -- Signifikant lengde på shopping og bank, ikke signifikant for smartphone
        -- Signifikant compleksitet på shopping, smartphone og bank
        -- Kan hende at erfaring med IT og sikkerhet påvirker tallene her siden det sikkert er mange menn som har erfaring i dette datasettet.
        -- Females har en høyere andel med mønstre av lengde 4 og en lavere mengde mønstre av lengde 9
            * Ytterverdiene varierer mes 5-8 er tilnærmet likt.
        -- Ingen kvinner som oppnåde max complexity score i datasettet på noen av pattern typene. 
        -- 4 av overlappene er laget av kvinner, resten av menn. 
        -- Veldig lav avg. intersection score!

    \section{Handedness and choice in graphical passwords}
        

    \section{Experience with IT and Security and choice in graphical passwords}
        -- Signifikante resultater for kompleksitet for shopping and bank, Ikke signifikant for smartphone
        -- Significant length for bank


--------------------------



    \section{Choice in graphical passwords and context of use}
        -- Kan se ut som om brukere gjør en form for risikovurdering.
        -- Personer velger å lage lengre passord for bank enn for shopping account og smartphone
        -- Hvorfor vil personer velge et sterkere låsemønster for en shopping account enn for smarttelefon? 
             Kan det ha noe med at personer synes det er stress å skrive inn lange møsntre for mobile enheter?
             Forstår man at det potensielt er skummelere å miste informasjon på en mobil enn på en shopping account?

        -- Related work:
            * Correlation between the use of security features and risk perception 
                - 33\% ser på bruken av låsemønstre som irriterende
                    Annet studie rapporterte over 48\% som syntes at låsemekanismer var irriterende, men samtidig
                    sa over 95\% at de agreed or fully agreed at de likte tanken på at mobilen deres var beskyttet. 
                - 26\% mente at ingen andre kunne ha nytte av informasjonen som lå på telefonen
                - 29\% låser ikke mobiltelefonen. (Muligens blitt litte bedre etter at fingerprint kom?)
                - Vi bruker 2.9\% av tiden vår som vi bruker på mobilen i snitt (9\% worst case) på å låse opp telefonen. 
                - 1/3 personer i et userstudy har mistet mobilen. 
                - Et studie rapportere at alle personene de snakket med hadde mail automatisk logget på, mens 31\% av dem brukte ikke skjermlås.

        -- Se hvordan den visuelle kompleksiteten endrer seg med konteksten: bank medfører at brukere prøver hardere fordi avg overlap og intersections øker. 

    \section{Similarities in the choice of graphical passwords in the entire population}
        -- Blir her mye sammenligning med annen forskning, mens de andre punktene har jeg ikke så mye å sammenligne med fordi det ikke er så mange som har gjor det før.
