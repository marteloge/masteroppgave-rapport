% !TEX root = ../main.tex
\chapter{Conclusion and Future Work}\label{chap:conclusion}
  
  This chapter includes the conclusion, as well as proposing ideas for future work. 

  Section \ref{sec:conclusion} is presenting the conclusion for the hypothesis as well as a conclusion for the research questions. Section \ref{sec:futureWork} purposes three ideas for future work based on the results from this research. 

  \clearpage
  \section{Conclusion}\label{sec:conclusion}
    
    The hypotheses tested in this research is whether users choice of graphical passwords is influenced by the human properties of the user. The hypotheses are the following:

    {\renewcommand\labelitemi{}
          \begin{itemize}
            \item \texttt{$H_{0}$: Human properties have no influence on a user's choice of \\graphical passwords}
            \item \texttt{$H_{1}$: A user's choice of graphical passwords is influenced by \\the human properties of the user}
          \end{itemize}
        }

    The human properties included in the results is age, gender, handedness, the users experience with IT and security, reading/writing direction and hand size. The results show that it is not possible to either accept or reject the hypotheses based on the data collected. 
    First, the experiment has not tested enough human properties, which means that the hypothesis may have been too broad for this experiment. To be able to answer this hypothesis, more data and properties need to be analyzed. Second, some properties had to be ignored due to poor data quality and challenges with the composition of the population.

    Even though the hypothesis nor can be accepted or rejected, the results indicate that there is a significant difference in patterns created by the different user types. A conclusion of the research questions are stated below. 


    \subsubsection*{The choice in graphical passwords and age}
    The results show statistical significant results proving that age impacts the choice in patterns. 
    The result shows that respondents under 25 years creates longer patterns with a higher visual complexity than the respondents having an age of 25 or higher. 

    \subsubsection*{The choice in graphical passwords and Gender}

    \subsubsection*{The choice in graphical passwords and Handedness}
    The results in this reearch provides no evidence for choice in graphical password being influenced by handedness.

    \subsubsection*{The choice in graphical passwords and Experience with IT and security}

    \subsubsection*{The choice in graphical passwords and reading/writing orientation}
    The collected data did not provide a sufficient amount of data for being able to make any conclusions of whether choice in graphical passwords are influenced by the reading and writing orientation.

    \subsubsection*{The choice in graphical passwords and size of hand}

  ----------

    \subsubsection*{The choice in graphical passwords and context of use}

    \subsubsection*{The choice in graphical passwords in the entire population}

    %- Konklusjon: her må du gjenta hypotesen. Nå er konklusjonen noe vag og det er akkurat det den ikke skal være. Vær konkret. Si: dette var hypotesen. Dette var de menneskelige egenskapene vi testet for. Resultatene er dessverre ikke av en slik art at vi kan bekrefte eller forkaste hypotesen. Det er fordi 1) vi har ikke testet alle mulige menneskelig egenskaper (mao var hypotesen for vid i forhold til eksperimentet) 2) av de vi testet var det ikke alle vi kunne konkludere på bla pgs utfordringer med sammensetningen av populasjonen. Men: selv om du ikke kan konkludere på hovedhypotesen, så kan du det på forskningsspørsmålene. Så trekk de frem igjen her og gå gjennom hver enkelt og konkluder.

    %... looking at predictable behaviour corresponding to different user types
    %-- Ikke nok data til å akseptere eller forkaste, men det er indikasjoner på at noen egenskaper påvirker valg av grafiske låsemønstre.
    %-- Resultatene viser generell predictable behaviour as seen in other publishe research.

    %The properties not used are still interesting properties to look at and will be purposed for future research.

  \clearpage
  \section{Future Work}\label{sec:futureWork}

    This section provides a list of three suggestions for future research based on the results in this research.

    \subsection{Reading and Writing Orientation and Choice of Graphical Passwords}
      Instead of observing a correspondence between handedness and selection of graphical password, this study found that handedness have any impact. Since both left- and right-handed started their patterns on the left side instead of starting at a different side as first predicted, reading and writing orientation looks even more promising than previously thought. Unfortunately, this project did not manage to collect enough data from respondents having an another reading and writing direction than from left-to-right. Reading and writing orientation looks like a human property having a potential for giving positive results when looking at people's choice in graphical passwords. 

    \subsection{The Use of Statistical Attack Models in Forensics}
      This study started with the far-fetched idea of "tell me who you are and I will tell you your lock pattern". This idea is difficult to solve in practice. For being able to achieve the described idea, an attack model needs to be built. Previous research have created a statistical attack model, also known as a {\it Markov Model}, only using the patterns \cite{Uellenbeck}. By applying knowledge of how patterns are created by the different user types, it is believed that the guessing rate of the model can be improved. 

      The described model are requested from a security intelligence group in a country whereas the name need to stay unmentioned. Such attack model can assist security intelligences in forensic cases, as mobile devices can provide important data for solving such cases. When needing to catch a criminal, time and information is crucial. Android Pattern Lock have in particular been reported as a problem where it currently lacked research. For future research, it is possible to use the results of this research for building models to predict patterns with a higher success rate than is being possible today. 

    \subsection{Exploring Other Graphical Password Schemes}
      The Android Lock Pattern are one of the few locking mechanisms available on mobile devises. When reviewing the literature, there are lacking studies looking at human properties and choice of graphical passwords. This study looked in particular at the Android Lock Pattern. It would be interesting to see if the results in this research were found across different locking mechanisms. Such knowledge can be used to get a better understanding of how people handle security on mobile devices. How people manage security can provide insightful information for building locking mechanism assisting users to create more secure passwords.
    