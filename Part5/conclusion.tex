% !TEX root = ../main.tex
\chapter{Conclusion and Future Work}\label{chap:conclusion}
  

  Section \ref{sec:conclusion} presents the conclusion for the hypothesis ond the research questions. Section \ref{sec:futureWork} proposes ideas for future work based on the results from this research. 

  \clearpage
  \section{Conclusion}\label{sec:conclusion}
    

The hypotheses in this research tests whether users' choice of graphical passwords is influenced by the human properties of the user. The hypotheses are the following:

    {\renewcommand\labelitemi{}
          \begin{itemize}
            \item \texttt{$H_{0}$: Human properties have no influence on a user's choice of \\graphical passwords}
            \item \texttt{$H_{1}$: A user's choice of graphical passwords is influenced by \\the human properties of the user}
          \end{itemize}
        }

The human properties included in this research are age, gender, handedness, the user's experience with IT and security, reading/writing direction and hand size. 

Unfortunately, the alternate hypothesis was not possible to either accept or reject based on the data collected. First, enough human properties was not collected, which means that the hypothesis may have been too broad for the experiment. To be able to answer this hypothesis, more data and properties need to be analyzed. Second, some properties had to be ignored due to poor data quality and challenges with the population composition. 

Even though the hypothesis cannot be accepted or rejected, the results show a significant difference in patterns created by various user types. A conclusion of the research questions is enumerated below. 

\subsubsection*{RQ1 - The choice of graphical passwords and age}
The results confirm that the respondents under 25 years creates longer and more complex patterns than people older than 25.

\subsubsection*{RQ2 - The choice of graphical passwords and gender}
There was a significant difference in the patterns created by male and female respondents, where male respondents created longer and more complex patterns. 

\subsubsection*{RQ3 - The choice of graphical passwords and handedness}
The results provide no evidence of a correlation between handedness and choice of graphical passwords. The length and visual complexity of the patterns created by left- and right-handed respondent were not significantly different. 

\subsubsection*{RQ4 - The choice of graphical passwords and experience with IT and security}
The data shows a significant difference in pattern length and visual complexity between people with high IT and security experience, and those with little experience. The patterns created by experienced respondents were longer and had a higher visual complexity than the patterns created by inexperienced respondents.

\subsubsection*{RQ5 - The choice of graphical passwords and reading/writing orientation}
The collected data was not sufficient to make any conclusions if choice in graphical passwords is influenced by reading and writing orientation.

\subsubsection*{RQ6 - The choice of graphical passwords and size of hand}
The collected hand size data could not be used because the size classification could not be verified. Therefore,no conclusions about hand size and choice of patterns were made.

\subsubsection*{RQ7 - The choice of graphical passwords of the entire population}
The results show predictable behaviour when looking at the entire population. There is a bias towards the selection of starting node. In addition, a significant number of patterns correspond to a letter in the alphabet. The top 100 patterns in the dataset constituted 42\% of the collected patterns, indicating that users select similar patterns. 

\subsubsection*{RQ8 - The choice of graphical passwords and context of use}
The results show that there is a difference in patterns created for the various pattern types. The respondents create longer and visually more complex patterns for banking accounts. The patterns created for smartphones had the lowest average length as well as being less visually complex.

  \clearpage
  \section{Future Work}\label{sec:futureWork}

    This section provides a list of three suggestions for future research based on the results in this research.

    \subsection{Reading and Writing Orientation and Choice of Graphical Passwords}
      Instead of observing a correspondence between handedness and selection of graphical password, this study found that handedness have any impact. Since both left- and right-handed started their patterns on the left side instead of starting at a different side as first predicted, reading and writing orientation looks even more promising than previously thought. Unfortunately, this project did not manage to collect enough data from respondents having an another reading and writing direction than from left-to-right. Reading and writing orientation looks like a human property having a potential for giving positive results when looking at people's choice in graphical passwords. 

    \subsection{The Use of Statistical Attack Models in Forensics}
      This study started with the far-fetched idea of "tell me who you are and I will tell you your lock pattern". This idea is difficult to solve in practice. For being able to achieve the described idea, an attack model needs to be built. Previous research have created a statistical attack model, also known as a {\it Markov Model}, only using the patterns \cite{Uellenbeck}. By applying knowledge of how patterns are created by the different user types, it is believed that the guessing rate of the model can be improved. 

      The described model are requested from a security intelligence group in a country whereas the name need to stay unmentioned. Such attack model can assist security intelligences in forensic cases, as mobile devices can provide important data for solving such cases. When needing to catch a criminal, time and information is crucial. Android Pattern Lock have in particular been reported as a problem where it currently lacked research. For future research, it is possible to use the results of this research for building models to predict patterns with a higher success rate than is being possible today. 

    \subsection{Exploring Other Graphical Password Schemes}
      The Android Lock Pattern are one of the few locking mechanisms available on mobile devises. When reviewing the literature, there are lacking studies looking at human properties and choice of graphical passwords. This study looked in particular at the Android Lock Pattern. It would be interesting to see if the results in this research were found across different locking mechanisms. Such knowledge can be used to get a better understanding of how people handle security on mobile devices. How people manage security can provide insightful information for building locking mechanism assisting users to create more secure passwords.
    
