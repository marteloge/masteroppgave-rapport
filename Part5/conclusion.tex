% !TEX root = ../main.tex
\chapter{Conclusion and Future Work}\label{chap:conclusion}
  

  \clearpage
  \section{Conclusion}
    
    The hypothesis tested is to check whether users choice of graphical passwords are influenced by the human properties of the user. The human properties that was included in the experiment was age, gender, handedness, and the users experience with IT and security. 
    Unfortunately, it is not possible to either accept or reject the hypothesis based on the data collected. First, the experiment have not tested enough human properties, which means that the hypothesis may have been to broad for this experiment. To be able to answer this hypothesis, more data and properties need to be analyzed. Second, some properties had to be ignored due to poor data quality and challenges with the composition of the population.

    Even though the hypothesis nor can be accepted or rejected, the results indicate that there is a significant difference in patterns created by the different user types. 


    Age: 

    Gender:

    Handedness:

    Experience with IT and security:

    %- Konklusjon: her må du gjenta hypotesen. Nå er konklusjonen noe vag og det er akkurat det den ikke skal være. Vær konkret. Si: dette var hypotesen. Dette var de menneskelige egenskapene vi testet for. Resultatene er dessverre ikke av en slik art at vi kan bekrefte eller forkaste hypotesen. Det er fordi 1) vi har ikke testet alle mulige menneskelig egenskaper (mao var hypotesen for vid i forhold til eksperimentet) 2) av de vi testet var det ikke alle vi kunne konkludere på bla pgs utfordringer med sammensetningen av populasjonen. Men: selv om du ikke kan konkludere på hovedhypotesen, så kan du det på forskningsspørsmålene. Så trekk de frem igjen her og gå gjennom hver enkelt og konkluder.

    %... looking at predictable behaviour corresponding to different user types
    %-- Ikke nok data til å akseptere eller forkaste, men det er indikasjoner på at noen egenskaper påvirker valg av grafiske låsemønstre.
    %-- Resultatene viser generell predictable behaviour as seen in other publishe research.

    %The properties not used are still interesting properties to look at and will be purposed for future research.

  \clearpage
  \section{Future Work}

    Instead of observing a correspondence between handedness and selection of starting node, it was observed that handedness did not impact the selection in starting node. The results and published research gave an indication that reading and writing direction can be a property that is more promised to give any result. This study tried to collect the reading and writing direction, but did not manage to obtain a data set that was sufficient enough. 

    This study started with the far fetched idea of "tell me who you are and I will tell you your lock pattern". This idea is, of course, an overall idea that in practice are hard to reach. It is requested from other researchers to conduct more global research including other parameters that may be of significance, such as the users educational level, geographical location and other demographic features \cite{Andriotis}. This was not managed in this project, but it is still a need research to be able to answer the hypothesis stated in this research. I hope someone can use this thesis as an inspiration for continuing on the work that is started. 

    The Android Lock Pattern are one of the few locking mechanisms available on mobile devises. When reviewing the literature, there lacked in general studies looking at human properties and selection of a password. This study looked in particular at the Android Lock Pattern. It would be interesting to see if the same behavior is found across different locking mechanisms. Such knowledge can be used to get a better understanding of how people cope with security on mobile devices to build locking mechanism supporting users to obtain higher security. Text-based passwords and PIN codes have been used for many years with a lot of educational advice for how to create secure passwords. Educational advice on how to secure the mobile device are lacking. As mobile devices get more involved in our daily lives, it becomes important to learn people the consequence of not respecting the need for high security. 

    


    %- Does reading and writing impact the way we create graphical passwords?
    %- Se mer på folks risikovurdering
      %* Skummelt at så mange vurderer mobil som minst viktig
      %* Skummelt å se at så mange veldiger lite visuelt komplekse mønstre
      %* Sammenligne studier av ALP med andre låsemekanismer.
      %* Finne alternative låsemekanismer?
