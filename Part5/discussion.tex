% !TEX root = ../main.tex
\chapter{Discussion}\label{chap:discussion}

  This chapter is a discussion of the results presented in Chapter \ref{chap:results}. The first six sections corresponds to the first six research questions from 
  Chapter \ref{chap:introduction} focusing on choice in passwords and human properties. Section \ref{sec:discussionEntirePopulation} are 
  discussing the results when looking at the entire population. Section \ref{sec:discussionContext} discusses the results from 
  a context of use perspecitive. The last section is a disussion of the limitations of this research.

  \clearpage

  \section{Age and Choice of Graphical Passwords}

    When testing the significance of the length and complexity for different age groups, the test distinguished two groups; under and over 25 years. The tests revealed that there is a significant difference in pattern length and visual complexity for patterns created for shopping accounts and banking accounts. The respondents under 25 created longer patterns as well as patterns with a higher visual complexity for banking accounts and shopping accounts. There was no significant difference in length and the complexity of patterns created for smartphones. 

    Before starting to analyze the results, it was not expected that the respondents under 25 years would create longer and more complex patterns than the respondents over 25. It was rather being expected that the respondents over 25 years created longer and more complex patterns as they might, for example, use the mobile device for work purposes where a strict security policy is often required. A factor that may cause the difference in complexity and pattern length is that the respondents under 25 have owned a mobile device when growing up. Knowing how to use a smartphone might make younger respondents more prepared for creating stronger passwords as they might be more familiar with the utilization of a smartphone. As far as this research is familiar with, no other research has been found regarding the selection of password particularly looking at age.

  \section{Gender and Choice of Graphical Passwords}

    By comparing the patterns created by male and female participants, there was a significant difference in patterns created. The results revealed a significant difference in patterns created for shopping account and banking accounts when testing the pattern length. When testing the visual complexity of the patterns, there was a significant difference in the created patterns for all pattern types. The results prove as evidence that male participants create longer patterns with higher visual complexity compared to the patterns created by female respondents.

    By studying the choice of length, female participants had a higher frequency of patterns of length four and at the same time had a lower frequency of patterns of length nine as opposed to male participants. When looking at patterns with length five up to eight, both genders have the same frequency. There is therefore a difference in how frequently male and female participants select the minimum and the maximum pattern length. One factor that may be involved is that many of the male participants are experienced with IT and security while fewer of the female respondents have the same experience. 

    By studying the visual complexity of the patterns created by the two genders, it is observed that none of the female participants managed to create a pattern with the maximum score in the dataset.  The cause of female participants choosing less secure patterns might be biased by the number of male participants with a background in IT and Security or that there are more male participants in the dataset. As far as this research is familiar with, no other research has been found focusing on the selection of passwords based on gender. One research group was investigating the security of the PassFace scheme where it was observed that users tended to choose faces that they liked or could compare themselves to \cite{Davis}. By knowing the gender, they managed to perform a dictionary attack to guess user selected passwords. If the gender of the user was known as male, then 10\% of the passwords could easily be guessed on the first or second attempt. 

  \section{Handedness and Choice of Graphical Passwords}
    
    Handedness is a biological characteristic that relates to how people interact with physical objects. In general, the majority of the population have the property of being right-handed while only 12\% of the population are left-handed. The results do not show any significant results indicating any differences in length and visual complexity for patterns created by left- and right-handed respondents. Thus, there is no statistical evidence for graphical passwords being influenced by the handedness of the creator.

    When selecting handedness as a human property in this study, it was believed that handedness and the way that respondents interacted with the touch screen would impact the selection of starting node. Based on the result, two main ways of interacting with a touch screen for unlocking a smartphone were observed. The first, and most common way, is to use one hand for holding and interacting with the phone. The other way is using one hand for holding the phone while using the forefinger on the other hand for interacting with the screen. 85\% and 53\% of the right- and left-handed respondents, respectively, used the two described methods for interaction. The results confirm that left-handed respondents do not have a main way of interacting with a smartphone as for right-handed respondents. An explanation for the difference in the physical behavior could be a historical impact where left-handed children were forced to write with their right hand.

    44\% of the respondents started their patterns in the upper left corner, wheras 73\% of the patterns either started in the the upper left corner, upper right corner or the bottom left corner. Because of the biased selection of starting node, and the different ways of interacting with the touch screen, it was believed that there would be a difference in the selection of starting node by looking at handedness. Since the majority of respondents was right-handed, it was assumed that this was the factor causing the high frequency of patterns starting in the upper left corner. Therefore, a higher frequency of patterns starting in the upper right corner as a cause of handedness and physical interaction was expected when regarding left-handed respondents. However, the results revealed the opposite, where the majority of the patterns started on the left side of the grid. The majority of the left-handed respondents would rather start on the left side than the right side as originally assumed.

    An explanation for the unexpected behavior is that right-handed and left-handed respondents prefer to start the patterns at the same nodes. The cause of the same behavior need to be looked further into, but studies exploring how memory retrieval and storage works can be a good start. Researchers have found that people prefer to scan information and store information in the short memory in a better way when scanning according to preferred reading and writing direction \cite{Chan}. As a result of the way we scan information, it may cause users to start creating the patterns in the same way we read and write to be able to remember the patterns we create. The results do not imply that handedness have any impact on the way we create out patterns, rather a stronger indication that reading and writing orientation have a stronger impact on the way we create our patterns. In this study, less than 2\% of the population had a different reading and writing orientation than the majority, e.g. left-to-right. Thus, it is not possible to use the collected data to make any conclusions about the impact on the choice of patterns based on the reading and writing orientation.

  \section{Experience with IT and Security and Choice of Graphical Passwords}
    
    The result in this research shows that there is a significant difference in the patterns created by experienced and inexperienced respondents. There is a significant difference in the length of patterns created for banking accounts, as well as a significant difference in the visual complexity for all pattern types. 

    If a person acknowledges that they have experience with IT and security, it is likely that they know the risks of selecting short passwords with low complexity that are easily guessed. The results indicated that the respondents with experience with IT and security are better with creating patterns, but there is still a low average complexity score and a high frequency of patterns created with the minimum length. Not a single respondent created a pattern with the highest complexity score for smartphones; the patterns created for smartphones were the patterns with the lowest average complexity score and length, regardless of the IT experience of the creator.

    The results shows that even experienced respondents do not create strong patterns. This study does not know why experienced respondents create short passwords when they probably know the consequence. One explanation is that users in general are not able to create and remember long and complex patterns. 

  \section{Reading and Writing Orientation and Choice in Graphical Passwords}
    As stated, the number of participants with a different reading and writing orientation than from left-to-right were too low, and thus cannot be used for making any conclusions. When looking at choice in graphical passwords and handedness, it was observed that there was no significant difference in the patterns created by left- and right-handed respondents. This observation gave a stronger indication of reading and writing orientation having an impact on the choice of graphical password. The reading and writing direction is proposed for future reasearch. 

  \section{Hand Size and Choice in Graphical Passwords}
    The data collected for the hand size were not used as a cause of being difficult to correctly verify the selected hand size. From the start, the property was known to have a subjective form, meaning that there was a risk of this happening. It is not known how the collection of hand size could be done differently, as this study selected the best-known approach. If wanting to look further into the choice of patterns and the size of the hand, it is probably a better idea to set up an experiment for being able to correctly measure the size. 

  \section{Similarities in the Choice of Graphical Passwords in the Entire Population}\label{sec:discussionEntirePopulation}

    This section looks at the results found when looking at the entire population. 

    \subsection{Pattern Creation Time and Length}
      The average pattern length for the entire population varied according to the pattern type. Patterns created for banking accounts had the highest average length of 5.92 while the patterns created for smartphones had the lowest average length of 5.40. Research have reported that the average length of patterns created for smartphones is 5.63 \cite{Uellenbeck}, which is close to the average length for smartphones observed in this study. 

      There are a different number of combinations of patterns with different length, but patterns of length eight seem to occur less often. For all pattern types, the probability of a pattern of length 8 to be selected is only 4-5\%. The cause of patterns with length 8 occurring less often then than patterns having a length of 7 or 9 nodes is not found. Patterns having a long length do not occur at the same frequency as the short patterns, but the patterns of length 8 are almost absent in some cases. There are 140.000 patterns of length eight, making the probability of selecting a password of length 8 high given a uniform selection. From a security perspective, this can reduce the number of combinations with roughly 140.000 combinations because of the low probability of users creating a pattern of length eight. Any reduction in the number of likely combinations is a violation of the security of the password scheme.

      Pattern creation time can tell a lot about the validity of the dataset. Respondents experienced with the Android Unlock Pattern had different reaction time when creating the patterns for a smartphone. The time used for creating patterns for a shopping account and bank account was about the same. The difference in creation time between respondents experienced and inexperienced with ALP was 1.2 seconds. The different reaction time can be a result of many respondents having shared their actual lock pattern, or an another pattern known to the respondent. This result indicates that the dataset possessed includes patterns representative for patterns used in "the wild". 

      As the result shows, people creates on average short patterns, especially short patterns are observed created for smartphones. From a security perspective, a smartphone could cause loss of huge amounts of sensitive information compared to a shopping account. As reported by researchers, getting access to a smartphone automatically logged into an email account can give access to sensitive information like SSN, Bank Account Number, Email Password and Home Address \cite{Egelman}
      Selecting short patterns can be seen as a form of bad risk assessment because it is a risk the possibility of losing sensitive information. The cause low priority of a secure password mechanism on mobile devices may be influenced by the trade-off between security and usability. Many users do not want to spend more time than needed for typing a password. Because of the rapid use of our mobile device, it makes users spend a lot of the when interacting with the phone to unlocking the screen. Research have reported that an average user will use about 2.9\% of the spend on interacting with their mobile phone to gaining access by unlocking the screen \cite{habits3}.

      The average pattern length, regardless of pattern type, seems to be low. One factor might be that the respondents had to retype the pattern they selected, indicating that people are not capable of remembering longer and more complex patterns. To be able to remember a password, regular use are required for permanently store the password in long-term memory. A pattern created for one-time use are stored in short-term memory, making it hard to recall a complex pattern just created due to how our short-term memory works \cite{DeAngeli}.

    \subsection{Visual Complexity}
      On average, the patterns collected for all pattern types were given a low complexity score. In the entire population, none of the participants managed to produce a pattern receiving the maximum complexity score of 46.8. The patterns receiving the highest average complexity score was patterns created for banking applications while patterns created for smartphones received the lowest complexity score. 

      Graphical passwords, especially the Android Pattern Lock, need to have their security evaluated on an extra dimension because of their graphical characteristics. Pattern locks can, for example, be easily captured by someone, accidentally or intentionally, by looking over someone's shoulder. Typing a long pattern will not help when the visual complexity is low. The length will help in terms of possible combinations, but will not necessarily avoid the possibility of someone capturing the pattern if the visual complexity is low.  For avoiding such capturing attacks, there is a variety of new schemes purposed for avoiding such capturing attacks \cite{Wiedenbeck, IPAS}. 

      Overlaps and intersections are some of the parameters used when evaluating the visual complexity of a pattern. Only 46 overlaps are registered in the dataset. The total number of patterns including overlaps may be lower because one pattern can have more than one overlap. One explanation for users not utilizing overlaps in their patterns is because it might not be very intuitive as a result of how Android explains how the scheme works. One rule commonly known is the rule of a node only being able to be selected one time. This rule does not make an overlap very intuitive to select because the pattern needs to go through an already selected node. The supported feature for creating overlaps should have been communicated to the users in a better way. The use of overlaps can assist users to create more visually complex and secure patterns. The average number of intersections are also low compared to the number of possible intersections in a pattern \cite{Sun}. When looking at all patterns collected, the patterns in the dataset had an average of only 0.27 intersections. 

      The observation of the low frequency of intersections and overlaps can be an indication of people finding it hard to remember visually complex patterns. This result can be used to reduce the likely password space because there is a low probability that people would create a pattern containing many intersections or any overlaps at all. This observation can be seen as a violation of the security of the password scheme because the number of possible combinations can be reduced. 
    
    \subsection{Association Elements}
      A unique aspect of graphical patterns is that they are visual and not just text or numbers with a semantic representation as in PIN codes and alphanumeric passwords. When creating a password, it is common to create a password that is associated with something you know to be able to remember and recall the password. In the dataset, 11.4\% of the collected patterns corresponded to a letter. The behavior of selecting Android Pattern Locks corresponding to association elements were also recognized in another user study \cite{Sun}. When looking at user-selected passwords, studies show that many users make use of graphical shapes, or objects, to support the process of remembering \cite{Weiss}. 

      In this project, only letters were considered as an association element. Other studies have found that users also uses numbers to form a pattern, indicating that there might be other association elements than letters used as association elements. An another research reordered the nodes in the Android Pattern scheme, whereas one of the rearrangements ended up having the same shape as a {\it Delta} \cite{Uellenbeck}. The participants recognized the element, making the majority of the participants creating a pattern corresponding to a {\it Delta}. 

      Whether some of the patterns are used as an association element, or just looking like one as a coincidence, is not possible to know. To be able to answer such question, a qualitative study are recommended to be able to ask people about their password selection strategy and whether the use of association elements are a used. 

    \subsection{Selection of Starting Node and Movement Patterns}
      One restriction of the Android Pattern Lock is that each node can only be selected once, making the selection of starting node crucial in terms of being able to guess a pattern. The result of this study shows an indication of bias towards the choice of starting node, whereas 44\% of the respondents started creating their patterns in the upper left corner. The second and third most common starting point was the upper-right corner and the bottom-left corner, which was selected as a starting node 15\% and 14\% of the time, respectively. The top three starting nodes thus constitutes 73\% of all patterns in the dataset. Given a uniform distribution, the probability of starting in any node should be 11\%, and the total probability of starting in the top three starting nodes would then be 33\%. The results show that users do not select their patterns uniformly.

      The predictable behavior when selecting the starting node corresponds to the results from another user study having the same amount of patterns starting in the upper-left corner \cite{Uellenbeck}. The study also reported having 73\% of all patterns starting in either the upper-left corner, upper-right corner or bottom-left corner, matching the observations in this study.

      The results do not only reveal that patterns are likely to start in the upper-left noder, patterns also include predictable movement sequences. Most of the patterns are straight lines close to the edges, whereas the subsequence \texttt{123} and \texttt{147} often occurs. It does not only often occur as a subsequence, but it is also one of the most commonly used ways of starting a pattern. This study reveals that 42\% of the created patterns are being found in the top 100 list of most frequently created patterns. In other words, by selecting a pattern from the top 100 list, there is a 42\% chance of a match. In the same list, the subsequences \texttt{123} and \texttt{147} are subsequence appearing most commonly. There is not found a specific explanation for this behaviour, but it occurs that the respondents preferred creating patterns close to the edges. Since there is a low frequency of patterns having intersections and overlaps, an explanation can probably be found by studying Gestalt and cognitive psychology. The study of psychology is outside the scope of this research, but the results strongly indicate that user-selected password are being biased as a cause of the visual appearance of the graphical scheme. The visual appearance is might causing users to select lines close to the edge as well as starting patterns in the corners.

  \section{Choice in Graphical Passwords and Context of Use}\label{sec:discussionContext}
    
    When looking at the selection of pattern it seems like the participants have performed a type of risk assessment. The patterns created do not appear to be created by chance because there is a difference in both average pattern length, creation time, and visual complexity for each pattern types.

    Sun et al. \cite{Sun} investigated the effect of using password meters for Android Lock Patterns to observe if it would assist users in creating stronger patterns. The result showed that the strength of the created patterns when using pattern meters resulted in higher complexity and strength, but also introduced a higher error rate when retyping the pattern. As seen in the result, the patterns created for smartphones have a lower length and complexity. Engelman et al. \cite{Egelman} reported that 33\% of the smartphone users were thinking about the locking mechanisms as too much of a hassle. Since higher complexity often can introduce a higher overhead in time used for unlocking the smartphone, it might be easier for users to select a pattern that are less complex to retype and remember. 

    Money is something all respondents are familiar with and understand the consequence of losing. When looking at the patterns created for smartphones, the results from this study indicated that users are not being aware of the consequences of choosing weak passwords on mobile devices. The respondents in this study on average selected short patterns for smartphones, with a low visual complexity. The described behavior might indicate that users do not understand the consequence if someone could easily guess the password and gain access to the device. A study reported that 26\% of the users did not think that someone would care about the information stored on their smartphone \cite{Egelman}. Other studies report otherwise, where only gaining access to for example an email exposed huge amounts of sensitive information \cite{Egelman}. 

    When setting up the survey with the different pattern types, it requires the respondents to make a risk assessment and prioritize what the most important; usability or security. Users might create a more complex pattern for a bank account because losing money is a tangible risk they might be familiar with. A study reported that users are typically more security conscious when they are aware of the need for such behavior \cite{Sasse}. At the same time, a study revealed that 48\% of the respondents thought that locking mechanisms were annoying. In the same study, 95\% of them agreed or fully agreed that they liked the idea of their smartphone being secured \cite{habits3}. 

  \section{Limitations}\label{sec:limitations}
    One of the limitations by utilizing a survey is that the accuracy and honesty of people's responses cannot be verified. The survey was selected for avoiding manual work, as well as for ethical concerns. When needing to keep respondents anonymous, there is no better or easier way for performing data collection and at the same time being able to verify the honesty of the responses. 

    Another limitation of using an online survey is that it is not possible to have control of who participates due to ethical concerns. When conducting research, it is often desired to have a control group to validate the quality of the collected data. To be able to use a control group one would have to ask people about their real patterns, which is not desirable for both ethical and security reasons. This experiment will only be able to see the choice in patterns based on the properties collected from the survey.

    Some of the data properties were subjective, meaning that it is hard to verify the correctness of the data. Som examples are hand size, screen size, and the question asking whether the respondent have a background in IT and security. The two first properties were not used further because the quality was too poor, but the experience of the users was used. There is no way of verifying that the answer is correct because the respondents are the ones deciding what "having a background in IT and security" means. 

    The research looks at the hypothesis to see whether human properties of the user impacts the choice in graphical passwords. It is important to mention that the result from this research is not valid for all graphical passwords. This research does not test for all human properties, making it possible to do further research on the choice of graphical password including other human properties than the ones contained in this research. 

