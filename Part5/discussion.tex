% !TEX root = ../main.tex
\chapter{Discussion}\label{chap:discussion}

	\begin{itemize}
		\item Personers oppfatning av sikkerhetsnivå. Er personer klare over hvor mye sensitiv data som ligger på en mobiltelefon? Hvordan kan forskningen her brukes til å vise at man må fokusere på høyere sikkerhetskrav på mobile enheter. Det nytter ikke i dag med sikre applikasjoner når mennesker er det svake leddet. Autinnlogging på applikasjoner er noe brukere krever, noe som gjør at ansvaret til en viss grad blir overført til brukeren.
		\item Er dataene som er samlet inn reelle? 
		\item Hvordan påvirker antall personer med bakgrunn innen it og sikkerhet dataene?
		\item Hvordan kan dataene brukes innen forensics?
		\item Kan man lage stereotyper av personlige karakteristikker og tilhørende mønster
		\item Er Android Patter Lock sikkert nok å bruke?
		\item Hvordan kan ALP sammenliges med andre låsemekanismer?
		\item Feilrater eller bøller som svarer tull --> Hvordan påvirker dette dataene?
		\item Hvorfor forekommer noen mønstre og sekvenser oftere enn andre?
		\item Hvordan kan dataene her brukes til forensics?
		\item Hvordan kan resultatene brukes til besvisstgjøring av å ha dårlige mønstre? Er ALP trygt å bruke?
		\item Løser grapfiske passord memorability, usability og security på samme tid? Noen av de sikreste mønstrene forekommer aldri, vil det si at det fortsatt er en sterk tradeoff mellom usability og security? Hvordan kan man løse dette?
		\item Hvordan brukes ALP ift PINs? Ser man de samme tendensene i begge mekanismene ?
		\item Bruken av tid korresponderer i noen grad også kanskje med lengden av mønster? Eller bruker brukerene ekstra tid på å tenke seg godt gjennom når de lager mønster på bank?
	\end{itemize}


\section{The credibility of the collected data} \label{sec:credibilityofdata}
	
	\todo[inline, color=green!60]{Har samlet inn en del låsemønstre som er i bruk eller som har blitt brukt + litt bakgrunnsinformasjon. Jobbes med og skal samle inn noen fler når jeg kommer tilbake til Trondheim. Skal bruke disse til å verifisere dat fra undersøkelsen.}

	\todo[inline, color=orange!80]{Burde jeg heller tatt med denne under diskusjon?}