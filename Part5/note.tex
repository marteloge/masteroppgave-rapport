- Italic font på navn


*** Discussion/Limitations
- Vet ikke om alle har svar ærlig
- Hypotesen er ikke gyldig for alle grafiske passord
- Tester bare et utvalg av mennskelige egenskaper

---> hente opp de to siste i future work.
---> utnytte limitations til noe positivt --> det jeg ikke fikk til kan andre studere videre.

The proposed hypothesis will have some limitations. {\it First}, the hypothesis do not include all potential human properties. If the results of the statistical test are accepting the alternative hypothesis, the results are not valid for proving a correlation between users choice in graphical passwords and human properties that are not selected in this research. {\it Second}, the hypothesis does neither prove that human choices based on the human properties is valid for all graphical password schemes. The selected graphical password scheme is the Android Unlock Pattern. Chapter \ref{chap:results} will include a detailed description of the human properties included in the experiment.

When conducting the experiment, it is not possible to have control of the participants due to ethical concerns. In an experiment is is often desired to have a control group to validate the collected data. To be able to use a control group I would have to ask people about their real patterns, that is not desired for security reasons. This experiment will only be able to see the cause and effect on the patterns collected in this research.

The collected data is also collected over the Internet, meaning that I as a researcher have no control on who people are and if answers provided by participants are committed honestly. As mentioned in the elaboration of the methodical approach, I should not be able to track whom participating in the study due to ethical concerns. It is neither desirable to take an pen-and-paper approach due to both ethical concerns and the time available for collecting an reasonable amount of data. The data is collected to be as representative as possible to the real world without violating any privacy concerns. 

\subsection{Limitations}

    The previous sections have presented the population. This section will describe which properties will be used further and what data that can not be used further in an analysis. 

    Training is mentioned as one pattern type, but will only be used for pattern specific analyis where it valid to use. There is no restrictions for the training patterns, making it unconsistent to use in an conclusion. 

    Figure \ref{fig:respondentsBasics} are showing the number of participants with different reading and writing orientation. The numbers will not be used further because there are not enough respondents with the orientation from top-to-bottom and right-to-left to be able to get any significant results.

    The handsize and screensize are two properties collected that are categorized as an subjective answer. They are further looked into in Section \ref{sec:classificationhandsizescreensize} for further validation.     

    The selected properties will be analysed looking at the time used for creating the patterns, the length of the created patterns, and an analysis of the patterns visual complexity. The described approach will also be used when looking at the entire population. 

    \subsection{Validity}

    \todo[inline, color=red!80]{Internal and external validaty}