% !TEX root = ../main.tex
\chapter{Discussion}\label{chap:discussion}

%What does the results show?
  %What does tey imply?
  %How do they relate to other reported research in the literature about your research topic?
  %Do your findings agree or disagree with those of the other researcherss or people authority?
  %What do you think is important in my results?
  %What relevance do they have for other researchers?
  %What relevance do they have for people in the real world beyond universities?

	\begin{itemize}
		\item Personers oppfatning av sikkerhetsnivå. Er personer klare over hvor mye sensitiv data som ligger på en mobiltelefon? Hvordan kan forskningen her brukes til å vise at man må fokusere på høyere sikkerhetskrav på mobile enheter. Det nytter ikke i dag med sikre applikasjoner når mennesker er det svake leddet. Autinnlogging på applikasjoner er noe brukere krever, noe som gjør at ansvaret til en viss grad blir overført til brukeren.
		\item Er dataene som er samlet inn reelle? 
		\item Hvordan påvirker antall personer med bakgrunn innen it og sikkerhet dataene?
		\item Hvordan kan dataene brukes innen forensics?
		\item Kan man lage stereotyper av personlige karakteristikker og tilhørende mønster
		\item Er Android Patter Lock sikkert nok å bruke?
		\item Hvordan kan ALP sammenliges med andre låsemekanismer?
		\item Feilrater eller bøller som svarer tull --> Hvordan påvirker dette dataene?
		\item Hvorfor forekommer noen mønstre og sekvenser oftere enn andre?
		\item Hvordan kan dataene her brukes til forensics?
		\item Hvordan kan resultatene brukes til besvisstgjøring av å ha dårlige mønstre? Er ALP trygt å bruke?
		\item Løser grapfiske passord memorability, usability og security på samme tid? Noen av de sikreste mønstrene forekommer aldri, vil det si at det fortsatt er en sterk tradeoff mellom usability og security? Hvordan kan man løse dette?
		\item Hvordan brukes ALP ift PINs? Ser man de samme tendensene i begge mekanismene ?
		
	\end{itemize}

	\begin{itemize}
		\item Personers oppfatning av sikkerhetsnivå. Er personer klare over hvor mye sensitiv data som ligger på en mobiltelefon? Hvordan kan forskningen her brukes til å vise at man må fokusere på høyere sikkerhetskrav på mobile enheter. Det nytter ikke i dag med sikre applikasjoner når mennesker er det svake leddet. Autinnlogging på applikasjoner er noe brukere krever, noe som gjør at ansvaret til en viss grad blir overført til brukeren.
		\item Er dataene som er samlet inn reelle? 
		\item Hvordan påvirker antall personer med bakgrunn innen it og sikkerhet dataene?
		\item Hvordan kan dataene brukes innen forensics?
		\item Kan man lage stereotyper av personlige karakteristikker og tilhørende mønster
		\item Er Android Patter Lock sikkert nok å bruke?
		\item Hvordan kan ALP sammenliges med andre låsemekanismer?
		\item Feilrater eller bøller som svarer tull --> Hvordan påvirker dette dataene?
		\item Hvorfor forekommer noen mønstre og sekvenser oftere enn andre?
		\item Hvordan kan dataene her brukes til forensics?
		\item Hvordan kan resultatene brukes til besvisstgjøring av å ha dårlige mønstre? Er ALP trygt å bruke?
		\item Løser grapfiske passord memorability, usability og security på samme tid? Noen av de sikreste mønstrene forekommer aldri, vil det si at det fortsatt er en sterk tradeoff mellom usability og security? Hvordan kan man løse dette?
		\item Hvordan brukes ALP ift PINs? Ser man de samme tendensene i begge mekanismene ?
		\item Hva kan man egentlig bruke de 3 mønstrene til? Man tenker at bank er noe man virkelig vil beskytte, så basert på hva en person har på mobilen kan man velge sikkerhetsnivå. Har man en telefon man er veldig redd for at noen skal komme inn på er det kanskje større sannsynlighet for å finne riktig mønster blant bank mønstrene enn for mobil som ofte har kortere lengde. 
		\item Bruken av tid korresponderer i noen grad også kanskje med lengden av mønster? Eller bruker brukerene ekstra tid på å tenke seg godt gjennom når de lager mønster på bank?
		\item Risikobasert analyse --> man vurderer tydeligvis mobil som den med lavest sikkerhet... Dette må da også inn i bakgrunn om dette skal med.
	\end{itemize}